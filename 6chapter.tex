\chapter{Conclusiones y perspectivas}\label{cap6}
\section{Conclusiones}

En este trabajo se ha encontrado que la radiaci\'{o}n de Hawking resonante en la configuraci\'{o}n de l\'{a}ser de agujeros negros es producto de un conjunto discreto de inestabilidades, tanto si el sistema es un condensado de Bose-Einstein (BECs) donde la radiaci\'{o}n son fonones, como si el sistema son pulsos de luz propag\'{a}ndose en una fibra \'{o}ptica donde la radiaci\'{o}n son fotones. Para llegar a este resultado es necesario aplicar la teor\'{i}a de inestabilidades al l\'{a}ser de agujeros negros.\\

Para cumplir este objetivo iniciamos mencionando las propiedades de un campo cu\'{a}ntico al  propagarse en un espacio-tiempo curvo como el generado por un agujero negro. Este tipo de curvaturas poseen una regi\'{o}n de no retorno que se define como horizonte de eventos y la radiaci\'{o}n que logra escapar de \'{e}sta se conoce como radiaci\'{o}n de Hawking \citep{Hawking1974}.\\

Posteriormente mostramos c\'{o}mo el trabajo de \cite{Unruh1981} abre la posiblidad de explorar la predici\'{o}n de Hawking en sistemas an\'{a}logos, donde se comprueba que la din\'{a}mica de una fluctuaci\'{o}n que se propaga en un fluido en movimiento es equivalente a la din\'{a}mica de un campo cu\'{a}ntico en espacio-tiempo curvo. En el fluido debe existir una regi\'{o}n que cumpla la definici\'{o}n de horizonte para que se genere el proceso de Hawking.\\

Desde el trabajo de Unruh, ha habido numerosas propuestas para probar por analog\'{i}a varias predicciones de gravitaci\'{o}n y cosmolog\'{i}a entre ellas la radiaci\'{o}n de Hawking utilizando experimentos de laboratorio en materia condensada y \'{o}ptica. Los sistemas m\'{a}s prometedores son BECs y pulsos en fibras \'{o}pticas ya que permiten la construcci\'{o}n de horizontes.\\

En presencia de dos horizontes, uno interno o agujero blanco y otro externo o agujero negro la radiaci\'{o}n de Hawking sufre un proceso de autoamplificaci\'{o}n si el campo es bos\'onico y la relaci\'{o}n de dispersi\'{o}n es an\'{o}mala en el caso de BECs o normal en el caso de fibras \'{o}pticas. La estimulaci\'{o}n de la radiaci\'{o}n de Hawking es una consecuencia del confinamiento de las part\'{i}culas con frecuencia negativa en la regi\'{o}n entre los horizontes y dicha regi\'{o}n funciona como una cavidad. Esta configuraci\'{o}n es conocida como l\'{a}ser de agujeros negros \citep{Corley1999}.\\

La relaci\'{o}n de dispersi\'{o}n an\'{o}mala dada en la ec. (\ref{ec:disper}) y que describe la din\'{a}mica de una fluctuaci\'{o}n en un BEC se encuentra linealizando la ecuaci\'{o}n de Gross-Pitaevkii (\ref{ec:GP2}), mientras la relaci\'{o}n de dispersi\'{o}n normal que siguen  fluctuaciones que se propagan  en una fibra \'{o}ptica se encuentra linealizando la ecuaci\'{o}n no lineal de Schr\"odinger (\ref{NLS}).\\


Analizando por teor\'{i}a de inestabilidades la din\'{a}mica de una fluctuaci\'{o}n en la configuraci\'{o}n de BHL seguiendo \cite{2018Bermudez}, en el cap\'{i}tulo \ref{cap3} se logra predecir que los modos resonantes en la cavidad son producto de un conjunto discreto de inestabilidades, i.e., las frecuencias con las que oscila cada modo de la cavidad son complejas, de la forma: $\omega=\omega_R+i\omega_I$ con $\omega_I>0$. Este modelo no solo logra obtener los mismos resultados cuando se analiza el sistema con un modelo simple dado en \cite{Leonhardt2007}, sino que adem\'{a}s predice que hay modos que logran propagarse fuera de los horizontes de manera similar a la radiaci\'{o}n de Hawking. Para frecuencias reales no se evidencia este comportamiento.\\

La idea de l\'{a}ser de agujero negro fue explotada inicialmente en sistemas ac\'{u}sticos como los BECs, pero \cite{Faccio2012} y \cite{GaonaReyes2017} logran expandir el modelo al caso \'{o}ptico que es la configuraci\'{o}n de OBHL. En este caso al analizar la ec. (\ref{ec:gaona}) por teor\'{i}a de inestabilidades en la configuraci\'{o}n de OBHL se encuentra que la radiaci\'{o}n resonante en la cavidad ser puede vista como el producto de un conjunto discreto de inestabilidades. Con el modelo de inestabilidades no solo se reproducen los resultados ya conocidos sino que adem\'{a}s  permite predecir nuevos resultados que no se obtienen con el modelo simple como es que la radiaci\'{o}n atrapada entre los horizontes que forman la cavidad puede tunelar y escapar de ella, similar al habitual comportamiento de la radiaci\'{o}n de Hawking, encontrado de forma an\'{a}loga en el caso de las inestabilidades en el sistema ac\'{u}stico.\\

Adem\'{a}s solo para condiciones muy espec\'{i}ficas se pueden encontrar modos para los cuales hay una mayor probabilidad de encontrar el campo dentro de la cavidad que fuera de ella, i.e., es posible confinar luz con luz en un medio no lineal. Este \'{u}ltimo  resultado concuerda a nivel cualitativo con los monomodos que se propagan en una gu\'{i}a de onda temporal expuesta en \cite{Plansinis2016}.\\

Los resultados expuestos aqu\'{i} reflejan que la construcci\'{o}n de horizontes en sistemas an\'{a}logos permite estudiar la radiaci\'{o}n de Hawking. Por \'{u}ltimo podemos resumir las diferencias que se encuentran entre el modelo ac\'{u}stico y  \'{o}ptico en la siguiente tabla

\begin{table}[htb]
\centering
\begin{tabular}{|r|l|l|}
\hline
Caracteristica & Ac\'{u}stico& \'{O}ptico \\
\hline \hline
Radiaci\'{o}n & Fonones&Fotones\\ \hline
Sistema & BECs&Fibras \'{o}pticas \\ \hline
Ecuaci\'{o}n a linealizar & Gross-Pitaevkii& No lineal de Schr\"odinger \\ \hline
Relaci\'{o}n de dispersi\'{o}n&An\'{o}mala&Normal\\ \hline
Velocidad en exterior& $v_1<c$&$\upsilon_1=-n_{g0}+\delta n_{\text{max}}$\\ \hline
Velocidad en interior& $v_2>c$&$\upsilon_2=-n_{g0}$\\ \hline
Frecuencia conservada&$\omega$&$\omega'$\\ \hline
Coordenadas&$(x,t)$&$(\tau,\zeta)$\\ \hline
\end{tabular}

\caption{Diferencias entre el l\'{a}ser de agujeros negros en los caso ac\'{u}stico y \'{o}ptico.}
\label{tabla:anchofijo}
\end{table}


\section{Perspectivas}
En este contexto, la radiaci\'{o}n de Hawking se produce cuando el vac\'{i}o cu\'{a}ntico es dispersado por un horizonte que mezcla modos de frecuencia positiva con modos de frecuencia negativa manteniendo la conservaci\'{o}n de la norma. Para ver esto se debe construir una matriz de dispersi\'{o}n y encontrar los modos de salida en t\'{e}rminos de los modos de entrada, estos modos se definen de acuerdo a su direcci\'{o}n de propagaci\'{o}n cerca del horizonte. Al calcular esta matriz de dispersi\'{o}n se obtendr\'{i}a la probabilidad de mezcla entre modos de frecuencias diferentes en cada horizonte, en otras palabras, se obtendr\'ian los coeficientes de transmisi\'{o}n y reflexi\'{o}n de los horizontes vistos como una interface. \\

Una variaci\'on que se le podr\'{i}a hacer al modelo es cambiar el perfil de velocidad $\upsilon(\tau)$ plano que trabajamos por un perfil secante hiperbolica que asemeje de forma m\'as realista las fronteras de la cavidad formada por los pulsos intensos. Se podr\'ia resolver el problema de forma num\'{e}rica y ver si los modos resultantes son comparables al modelo expuesto en este trabajo. 