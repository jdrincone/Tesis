\chapter{Introducci\'{o}n}\label{cap1}
La existencia de los agujeros negros fue una de las primeras predicciones sorprendentes y novedosas que surgieron de la teoría de gravitaci\'{o}n de Einstein. Aunque inicialmente fueron punto de controversia, en la actualidad, está bien establecida su existencia y su papel en la dinámica y evoluci\'{o}n del universo a gran escala \cite{bambi2018astrophysical}. Cuando se describen los campos cu\'{a}nticos sobre estas soluciones exóticas del espacio-tiempo se encuentran fen\'{o}menos fascinantes como la radiaci\'{o}n de Hawking.\\

S.W. Hawking en 1974  por medio de una derivaci\'{o}n semicl\'{a}sica predice que los agujeros negros emiten un flujo de radiaci\'{o}n t\'{e}rmica \citep{Hawking1974}, planteando con su resultado una serie de preguntas fundamentales con respecto a la interacción de la mecánica cuántica, la gravedad y la paradoja de la información en agujeros negros \citep{susskind2006paradox}.\\

Aunque el efecto Hawking se basa en principios físicos ampliamente aceptados, la observación astrofísica de esta radiación parece un hecho poco probable, al menos, en el futuro cercano. En 1981, W.G. Unruh propone que en lugar de observar la radiación de Hawking que emana un agujero negro astrofísico, se prodr\'{i}a verificar esta predici\'{o}n a partir de un modelo an\'{a}logo realizable en el laboratorio \citep{Unruh1981}. Esta propuesta indica que las perturbaciones que se crean  en un fluido en movimiento se propagan de manera s\'{i}milar a un campo escalar en un espacio-tiempo curvo, por lo cual si se imponen las mismas condiciones en este modelo que a la luz en el caso astrof\'{i}sico se tendr\'{i}a el mismo fen\'{o}meno de radiaci\'{o}n.\\

Desde el trabajo de Unruh, ha habido numerosas propuestas para probar por analogía varias predicciones de gravitaci\'{o}n y cosmología utilizando experimentos de laboratorios en materia condensada y \'{o}ptica. Estos emplean una gama de medios que incluyen helio líquido \citep{volovik2003universe}, \citep{jacobson1998event}, condensados de Bose-Einstein (BECs) \citep{zapata2011resonant}, \citep{2018Bermudez}, fibras \'{o}pticas \citep{philbin2008fiber}, \citep{drori2019observation}, materiales espintrónicos \citep{jannes2011hawking} y muchos m\'{a}s.\\

Las propuestas m\'{a}s prometedoras que se analiz\'{a}n brevemente en esta tesis son los an\'{a}logos ac\'{u}sticos donde los fonones se propagan en un BECs que fluye y los an\'{a}logos \'{o}pticos en donde pulsos de luz se mueven en un medio no lineal y dispersivo. Estos sistemas son los m\'{a}s prometedores porque permiten la construcci\'{o}n de un horizonte, un ingrediente escencial para que exista el efecto Hawking. Los horizontes se definen como el lugar donde la velocidad de una perturbaci\'{o}n es igual a la velocidad del flujo del fluido y son importantes por que dividen el espacio donde se propagan los campos en dos regiones.\\

En los BECs, el horizonte es creado cambiando la velocidad del fluido cu\'{a}ntico, tal que una perturbaci\'{o}n puede generarse en dos posibles regiones: una subs\'{o}nica donde la velocidad de la perturbaci\'{o}n es menor que la velocidad del fluido y otra supers\'{o}nica donde la velocidad de la perturbaci\'{o}n es mayor que la velocidad del fluido.\\

En los sistemas \'{o}pticos, la creaci\'{o}n de horizontes requiere m\'{a}s ingenio. La propagaci\'{o}n de la luz en medios no lineales como las fibras \'{o}pticas permite la existencia de solitones, pulsos de luz cuyo perfil de intensidad no cambia a lo largo del medio donde se propagan. La no linealidad del material permite la interacción entre estos pulsos y un paquete de ondas de menor intensidad (pulso de prueba) generando el análogo de horizonte de eventos. El solit\'{o}n interactúa con el pulso de prueba a trav\'{e}s del efecto Kerr, que da lugar a un peque\~{n}o cambio del \'{i}ndice de refracci\'{o}n que es proporcional a la intensidad del solit\'{o}n. Por tanto el pulso de prueba se alenta cuando incide sobre el solit\'{o}n y, si la diferencia de velocidad inicial entre el pulso y el solit\'{o}n es pequeña, este puede llevar al paquete de ondas a su propia velocidad, o lo que es lo mismo, un observador en el marco de referencia del solit\'{o}n, ver\'{i}a la se\~{n}al detenerse. Esto corresponde a un horizonte de eventos de agujero blanco: un punto más allá del cual la se\~{n}al no puede pasar.\\

Por otro lado los avances en esta línea han logrado la implementación de una configuración conocida como láser de agujeros negros (BHL) \citep{Corley1996}. Esta es una variaci\'{o}n del esquema básico, donde  el flujo de velocidad del condensado cambia en tres zonas: de una regi\'{o}n subs\'{o}nica a una supers\'{o}nica en un espacio finito y regresando de nuevo a una regi\'{o}n subs\'{o}nica. Para realizar el segundo cambio entre regiones se necesita la presencia de un segundo horizonte, el cual permite que la radiaci\'{o}n confinada entre los horizontes sufra un proceso de amplificaci\'{o}n a través de un mecanismo de resonancia similar al de un láser. El mismo fen\'{o}meno se puede encontrar en los an\'{a}logos \'{o}pticos llamado l\'{a}ser  de agujeros negros \'{o}ptico (OBHL)  \citep{Faccio2012}, \citep{GaonaReyes2017}.\\

El proceso de  amplificaci\'{o}n se encuentra al describir la din\'{a}mica de peque\~{n}as excitaciones que se producen en un medio en movimiento y estas pueden ser realizadas a partir de teor\'{i}a de inestabilidades \citep{Hydrodynamic}, la cual describe el proceso de amplificaci\'{o}n  de un campo permitiendo que las frecuencias en que este oscile sean complejas con su parte real positiva.\\

En este trabajo reproduciremos los calculos del trabajo \citep{2018Bermudez} donde se emplea la teor\'{i}a de inestabilidades para describir la radiaci\'{o}n resonante en la configuraci\'{o}n de un BHL en BECs. Adem\'{a}s con lo aprendido, se analizar\'{a} la configuraci\'{o}n del OBHL en el caso de un an\'{a}logo \'{o}ptico. En ambos casos se encuentra que la radiaci\'{o}n de Hawking  resonante es producto de un conjunto discreto de  inestabilidades, adem\'{a}s en el caso \'{o}ptico la configuraci\'{o}n OBHL permite confinar luz con luz un resultado descrito en el contexto de \'{o}ptica no lineal en \cite{Plansinis2016}. Pero para entender estos resultados iniciemos describiendo brevemente qu\'{e}  es la radiaci\'{o}n de Hawking.

\section{Radiaci\'{o}n de Hawking}

Poco después de que Einstein revelara su teoría  de gravitaci\'{o}n en 1915, los físicos se dieron cuenta que esto permitía la posibilidad de singularidades en el espacio-tiempo producto de un colapso gravitacional catastrófico. En lugares donde la densidad es extrema como el núcleo muerto de una estrella masiva, el espacio-tiempo podría ser arrastrado hacia una singularidad o agujero en el universo. Un límite en el espacio-tiempo llamado \textit{horizonte} de eventos marca la regi\'{o}n de no retorno, una vez que cualquier tipo de informaci\'{o}n f\'{i}sica pase este l\'{i}mite no se podr\'{a} saber m\'{a}s de ella, i.e, una vez formado el horizonte, no hay nada en la teoría de gravitaci\'{o}n que pueda dar informaci\'{o}n sobre lo que pasa en dicha regi\'{o}n. \\

Estos agujeros negros deberían existir para siempre, solo crecer y nunca encogerse. Pero en \cite{Hawking1974} y \cite{Hawking1975} se prueba que esta \'{u}ltima conclusi\'{o}n podr\'{i}a ser err\'{o}nea. Cuando se estudian las propiedades de un campo cu\'{a}ntico sobre un espacio-tiempo curvo, en part\'{i}cular, sobre la m\'{e}trica de Schwarzschild se encuentra que los agujeros negros pueden radiar part\'{i}culas con un espectro t\'{e}rmico, es decir, pueden evaporarse y desaparecer.\\

Para entender la afirmaci\'{o}n de Hawking debemos recurrir a algunos conceptos de teor\'{i}a cu\'{a}ntica de campos. El espacio est\'{a} lleno de campos cu\'{a}nticos, los cuales pueden oscilar en formas particulares. Estas oscilaciones o excitaciones del campo representan el n\'{u}mero de part\'{i}culas que permean el espacio, e.g., si no hay excitaciones el campo se encuentra en el estado de vac\'{i}o, el cual es el mismo para todo el espacio. Las posibles frecuencias de las oscilaciones del campo pueden ser positivas o negativas, siendo estas \'{u}ltimas interpretadas como part\'{i}culas que viajan atr\'{a}s en el tiempo o part\'{i}culas de antimateria \citep{aitchison2012gauge}. Todo lo anterior es cierto en un espacio-tiempo plano como el de Minkowski. Pero ¿qu\'{e} ocurre cuando un campo cu\'{a}ntico se propaga sobre un espacio-tiempo curvo? En part\'{i}cular uno descrito por la m\'{e}trica de Schwarzschild?\\

La curvatura causa problemas para definir un \'{u}nico vac\'{i}o del campo, los horizontes de eventos acotan el acceso a ciertos modos de oscilaci\'{o}n de campos cuánticos, perturbando as\'{i} el equilibrio que define el vacío. Como resultado, el concepto de partículas se vuelve ambiguo y su interpretación física se hace mucho más difícil \citep{dewitt1975quantum}, i.e., la curvatura indefine los campos en la proximidad de un horizonte. Surge la pregunta ¿c\'{o}mo definir un campo adecuadamente?, ¿qu\'{e} informaci\'{o}n f\'{i}sica se obtiene al lograr esto? Para responder estas preguntas se necesitaría una teor\'{i}a completa que una la relatividad general y la mecánica cuántica,
una teoría cu\'{a}ntica de la gravedad,
pero esto no existe a la fecha. A\'{u}n as\'{i}  Hawking encontr\'{o} una forma ingeniosa de dar respuesta a estas preguntas. Imaginó la trayectoria  de un rayo de luz en el espacio-tiempo (geod\'{e}sica nula) que se extiende desde el pasado lejano y contin\'{u}a hacia el futuro lejano. En su trayecto, la luz pasa un instante antes por una regi\'{o}n donde una estrella se convier\'{a} en agujero negro por colapso gravitacional . El rayo del que hablamos es el \'{u}ltimo en salir un instante antes de formarse el horizonte de eventos \citep{Hawking1975}.\\

Si el campo se encuentra en estado de vac\'{i}o antes de la formaci\'{o}n del horizonte, despu\'{e}s de la creaci\'{o}n de \'{e}ste, el mismo campo podr\'{a} estar en un estado excitado, es decir, contiene part\'{i}culas. Aunque no exista una teoría cu\'{a}ntica de la gravedad, la predici\'{o}n se logra realizar al comparar los estados de vac\'{i}o de las dos regiones asint\'{o}ticas al horizonte (pasado lejano y futuro lejano) donde el espacio-tiempo es plano y la teor\'{i}a cu\'{a}ntica de campos funciona sin problemas.\\

Matem\'{a}ticamente se trata de analizar las consecuencias de tener un campo escalar cu\'{a}ntico $\phi$ que obedece la ecuaci\'{o}n de onda en un espacio tiempo curvo

\begin{equation}\label{ec:onda}
\frac{1}{\sqrt{-g}}\partial_{\mu}\left(\sqrt{-g}g^{\mu \nu}\partial_{\nu}\right)\phi=0,
\end{equation} 
donde $g_{\mu \nu}$ es el tensor m\'{e}trico y $g$ su determinante. Para nuestro inter\'{e}s trabajamos con la m\'{e}trica de Schwarzchild
\begin{align}
 ds^2=g_{\mu \nu}dx^{\mu}dx^{\nu}=\left(1-\frac{r_s}{r}\right)dt_s^2-\left(1-\frac{r_s}{r}\right)^{-1}dr^2-r^2d\Omega^2,
\end{align}\label{eqn:metricsc}

$r_s=2GM/c^2$ es el radio de Schwarzschild que define el horizonte de eventos y $d\Omega^2=d\theta^2+\sin^2\theta d\phi^2$ es el elemento de \'{a}ngulo s\'{o}lido.\\
La singularidad en el horizonte de eventos $r=r_s$ es solo una singularidad de coordenadas en el sentido que puede ser removida bajo un cambio apropiado de estas. Por el contrario, la singularidad en $r=0$ es esencial o f\'{i}sica. Esta m\'{e}trica tiene como particularidad que $g_{\mu \nu}\rightarrow \eta_{\mu \nu}$ cuando $r/r_s\gg 1 $, i.e, la métrica de Schwarzschild $g_{\mu \nu}$ se aproxima de manera asint\'{o}tica a la métrica plana de Minkowski $\eta_{\mu \nu}=\text{diag}(-1,1,1,1)$.\\

Para un espacio-tiempo plano, el campo cu\'{a}ntico $\phi$ posee una descomposici\'{o}n en funci\'{o}n de los operadores de aniquilaci\'{o}n y creaci\'{o}n $(a,a^{\dagger}$) de la forma

 \begin{equation}\label{ec:descomposicion}
 \phi(x)= \int d\omega \left(f_{\omega}(x)a_{\omega}+f^*_{\omega}(x)a_{\omega}^{\dagger}\right),
 \end{equation}
 
siendo $f_{\omega}(x)$ los denominados \textit{modos de frecuencia positiva}  y $f_{\omega}^*(x)$ \textit{modos de frecuencia negativa}, que dan igualmente soluci\'{o}n a la ec. (\ref{ec:onda}). El operador  de destrucci\'{o}n $a_{\omega}$ permite definir su estado de vac\'{i}o $|0 \rangle$, tal que
 \begin{equation}
 a_{\omega}|0\rangle=0, \hspace{1cm} \forall \omega .
 \end{equation}
La descomposici\'{o}n del campo en la ec. (\ref{ec:descomposicion}) es v\'{a}lida en el l\'{i}mite asint\'{o}tico que es f\'{i}sicamente equivalente al pasado lejano mencionado.\\

En un espacio-tiempo curvo con m\'{e}trica $g_{\mu \nu}$ se seguir\'{a} cumpliendo la ec. (\ref{ec:onda}) pero no ser\'{a} posible tener una \'{u}nica descomposici\'{o}n espectral de la forma de la ec. (\ref{ec:descomposicion}) debido que las frecuencias positivas y negativas no tienen un significado invariante, cambian para cada observador. Tendr\'{i}amos entonces diferentes construcciones del espacio de Fock (espacio de estados para el campo) basados en diferentes nociones de frecuencia.\\

Una regi\'{o}n de inter\'{e}s donde el vac\'{i}o est\'{a} bien definido es en el futuro lejano, el cual tambi\'{e}n es un l\'{i}mite asint\'{o}tico y permite expresar el campo como
\begin{equation}
\phi(x)=\int d\omega'\left( g_{\omega'}(x)b_{\omega'}+g_{\omega'}^*(x)b_{\omega'}^{\dagger}\right),
\end{equation}
donde $g_{\omega'}(x)$ son nuevas funciones modales y $b_{\omega'}$ es el operador de aniquilaci\'{o}n que define el estado de vac\'{i}o $|0\rangle_{\text{out}}$ en la regi\'{o}n asint\'{o}tica del futuro lejano.\\

La comparaci\'{o}n entre los vac\'{i}os de ambas regiones se realiza a través de las transformaciones de Bogoliubov que permiten escribir las funciones modales y operadores de ambas regiones como
\begin{align}
g_{\omega'}&=\int d\omega\left(\alpha_{\omega \omega'}f_{\omega}+\beta_{\omega \omega'}f^*_{\omega}\right),\\
b_{\omega'}&=\int d\omega\left(\alpha^*_{\omega \omega'}a_{\omega}+\beta_{\omega \omega'}a_{\omega}^{\dagger}\right),
\end{align}
donde $\alpha^*_{\omega \omega'}$ y $\beta_{\omega \omega'}$ son conocidos como los \textit{coeficientes de Bogoliubov} y cumplen la normalizaci\'{o}n
\begin{equation}
 |\alpha_{\omega \omega'}|^2-|\beta_{\omega \omega'}|^2=1.
 \end{equation}
Por lo cual el n\'{u}mero de part\'{i}culas en todo el espacio que puede medir un observador en el futuro lejado $N_m=b_m^{\dagger}b_m$ en el en el estado vac\'{i}o del pasado lejano es:

\begin{equation}
\langle 0|N_{\omega}|0\rangle=\int d\omega'|\beta_{\omega \omega'}|^2,
\end{equation}
El t\'{e}rmino $\beta_{\omega \omega'}$ en la transformaci\'{o}n de Bogoliubov describe una mezcla de modos de frecuencia positiva y negativa producto de un proceso de dispersi\'{o}n que es generado por la existencia de un horizonte. El estado de vac\'{i}o que proviene del pasado lejado es dispersado por el horizonte generando fotones que quedan atrapados dentro del horizonte con frecuencia negativa y fotones que logran salir de \'{e}l con frecuencia positiva.\\

El n\'{u}mero de part\'{i}culas que llegan al dectector de un observador en el futuro lejano cuando se suprime la parte \'{a}ngular $d\Omega=0$ en la ec. (\ref{eqn:metricsc}) es  
\begin{equation}
\langle 0|N_{\omega}|0\rangle=\int d\omega'|\beta_{\omega \omega'}|^2=\frac{\delta(0)}{\exp\left( \frac{2\pi\omega}{\kappa}\right)-1},
\end{equation}
donde $\kappa\equiv c/2r_s=c^3/4GM$ es conocida como la gravedad superficial del agujero negro. El número de partículas diverge porque es el total de partículas en todo el espacio, esto es evidente en el t\'{e}rmino $\delta(0)$ que representa el volumen de todo el espacio, por lo cual se define una densidad de partículas $n_{\omega}$ tal que
\begin{equation}
\langle 0|N_{\omega}|0\rangle=n_{\omega}\delta(0),
\end{equation}
con lo cual,
\begin{equation}\label{ec:flujo}
n_{\omega}=\frac{1}{\exp\left( \frac{2\pi\omega}{\kappa}\right)-1}.
\end{equation}
Esta densidad de part\'{i}culas representa a su vez la distribuci\'{o}n de frecuencias de la radiaci\'{o}n, de hecho es un espectro de cuerpo negro el cual puede describirse a partir de un \'{u}nico par\'{a}metro que es la temperatura $T_H$
 \begin{align}
 T_H=\frac{\hbar \kappa}{2\pi k_B}=\frac{\hbar c^3}{8\pi G k_B M}\approx 6.169 \times 10^{-8} \text{K}\times\frac{M_{\bigodot }}{M} ,
 \end{align}\label{TH}

donde $k_B$ es la constante de Boltzmann. La temperatura es directamente proporcional a $\kappa$ o inversamente proporcional a la masa del agujero. Para simplificar el resultado y obtener una estimaci\'{o}n de \'{e}stos, la \'{u}ltima expresi\'{o}n permite conocer el valor de la temperatura en unidades de Kelvin para la radiaci\'{o}n que emite un agujero negro de $M$ masas solares.\\

¿Qu\'{e} valores de temperatura puede tomar la ec. (\ref{TH})? Por ejemplo, para el agujero negro supermasivo de la galaxia M87, del cual a la fecha se tiene la \'{u}nica imagen real y es el m\'{a}s conocido por el p\'{u}blico en general, tiene $6,500$ millones de masas solares, i.e., $M=6.5\times 10^{8} M_{\bigodot}$ por lo cual el valor de la temperatura de la radiaci\'{o}n que puede emitir es del orden de $10^{-17}\text{K}$. Esto est\'{a} $17$ \'{o}rdenes de magnitud por debajo de la temperatura del fondo c\'{o}smico de microondas (CMB) que es de $2.7\text{K}$. Adem\'{a}s, para agujeros negros estelares existe un l\'{i}mite m\'{i}nimo en su masa para que sean creados por colapso gravitacional, el límite de Tolman-Oppenheimer-Volkoff \citep{bombaci1996maximum}, el cual es de $\sim 3.0 M_{\bigodot}$ . Por tanto la cota superior en temperatura para agujeros negros estelares es de $10^{-8}\text{K}$ lo cual sigue siendo una temperatura extemadamente peque\~{n}a comparada con el CMB y pone de manifiesto una gran dificultad observacional para verificar este resultado en el caso astrof\'{i}sico.\\

Ahora, ¿qu\'{e} pasa si consideramos la parte angular de la m\'{e}trica?. En este caso el espectro de radiaci\'{o}n ya no es t\'{e}rmico, la ec. (\ref{ec:flujo}) viene apantallada por una funci\'{o}n que depende de la frecuencia, esta cantidad suele llamarse\textit{ factor gris} $\Gamma_{lm}(\omega)$ y en este caso ya no es posible caracterizar la radiaci\'{o}n de Hawking con un \'{u}nico par\'{a}metro \citep{mukhanov2007introduction}.\\


\section{Problema transplanckiano en la radiaci\'{o}n de Hawking}

El cálculo de la radiaci\'{o}n que sale de un agujero negro plantea un problema importante llamado \textit{problema transplanckiano} y se debe a que la radiaci\'{o}n de Hawking se basa en un aproximación semiclásica donde el campo escalar $\phi$ es cu\'{a}ntico mientras que el campo gravitacional $g_{\mu \nu}$ es tratado como un fondo clásico y fijo, generando un corrimiento exponencial al rojo en las frecuencias emitidas cerca del horizonte, i.e., la radiaci\'{o}n t\'{e}rmica que podr\'{i}a ser observada en el futuro lejano proviene de modos que salieron del pasado lejano con una longitud de onda extremadamente corta, mucho menor que la longitud de Planck  $l_p=\sqrt{\hbar G/c^3}=1.6\times 10^{-35}\text{m}$. Por debajo de la escala de Planck las teor\'{i}as con las cuales se describe la radiaci\'{o}n de Hawking dejan de tener validez.\\

Aunque la resolución completa de este problema requeriría una teoría completa de la gravedad cuántica, es interesante observar que modelos en gravedad an\'aloga permiten regularizar la radiaci\'{o}n de Hawking al estudiar la propagaci\'{o}n de un campo escalar en medios dispersivos \citep{jacobson1996introductory}.