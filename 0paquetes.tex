%paquetes
\usepackage[T1]{fontenc}
\usepackage[utf8]{inputenc}
\usepackage[spanish,mexico]{babel}
\usepackage{multirow} 
\usepackage{amsmath,bm}
\usepackage{graphicx,amssymb,amsmath,amsthm,amsfonts,mathtools,makeidx,bm,enumitem,float,relsize,empheq,lscape,emptypage,caption,bm,sectsty,xparse,cancel,mathrsfs,pifont}
\renewcommand\CancelColor{\color{red}} %cancelto color de la flecha
\usepackage[framemethod=default]{mdframed}
\usepackage[most]{tcolorbox}
\usepackage{parskip}
\captionsetup{margin=10pt,font=small,labelfont=bf}
\numberwithin{equation}{section}
\usepackage[lmargin=3.5cm,rmargin=2.5cm,tmargin=2.5cm,bmargin=2.5cm]{geometry}
\usepackage[square,colon]{natbib}
\bibliographystyle{plainnat}
\usepackage{fancyhdr}
    \pagestyle{fancy}                       					% Sets fancy header and footer
    \fancyfoot{}                            					% Delete current footer settings
    \renewcommand{\chaptermark}[1]{         					% Lower Case Chapter marker style
      \markboth{\chaptername\ \thechapter.\ #1}{}} 				%
    \renewcommand{\sectionmark}[1]{        						% Lower case Section marker style
      \markright{\thesection.\ #1}}        						%
    \fancyhead[LE,RO]{\textcolor{astral}{\bfseries\thepage}}    % Page number (boldface) in left on even
    \fancyhead[RE]{\textcolor{black}{\bfseries\leftmark}}      % Chapter in the right on even pages
    \fancyhead[LO]{\textcolor{black}{\bfseries\rightmark}}     % Section in the left on odd pages
    \renewcommand{\headrulewidth}{0.3pt}    					% Width of head rule
\subsubsectionfont{\color{cinves}}

%\usepackage{showkeys}%\usepackage{showlabels}
\usepackage{mathpazo}
\linespread{1.05}

\graphicspath{{Graphs1/}{Graphs2/}{Graphs3/}{Graphs4/}{Graphs5/}{Graphs6/}{Graphs7/}{Graphs8/}{Graphs9/}}
\hyphenation{Hei-sen-berg}

%Comandos para escritura rapida
\providecommand{\norm}[1]{\lVert#1\rVert}
\providecommand{\braket}[2]{\langle #1 | #2 \rangle}
\providecommand{\braketlr}[2]{\left\langle #1 \right|\left. #2 \right\rangle}
\providecommand{\ketbra}[2]{| #1 \rangle\langle #2 |}
\providecommand{\ketbralr}[2]{\left| #1 \right>\left< #2 \right|}
\providecommand{\ket}[1]{\lvert #1 \rangle}
\providecommand{\ketlr}[1]{\left| #1 \right>}
\providecommand{\bra}[1]{\langle #1 \lvert}
\providecommand{\conm}[2]{\left[ #1 , #2 \right]}
\providecommand{\avg}[1]{\left< #1 \right>}
\providecommand{\ti}[1]{\textit{#1}}
\providecommand{\tb}[1]{\textbf{#1}}
\providecommand{\tc}[1]{\textsc{#1}}
\providecommand{\mb}[1]{\mathbf{#1}}
\newcommand{\dd}{\text{d}}
%\newcommand{\mb}[1]{\mathbf{#1}}
\newcommand{\fa}{\forall\hspace{3pt}}
\newcommand{\intinf}{\int_{-\infty}^{\infty}}

\newcommand{\Q}{\mathbb{Q}}
\newcommand{\R}{\mathbb{R}}
\newcommand{\C}{\mathbb{C}}
\newcommand{\Z}{\mathbb{Z}}
\newcommand{\N}{\mathbb{N}}
\newcommand{\sD}{\mathscr{D}}
\newcommand{\so}{\mathfrak{so}}
\newcommand{\su}{\mathfrak{su}}
\def\cA{\mathcal{A}}
\def\cD{\mathcal{D}}
\def\cL{\mathcal{L}}
\def\cE{\mathcal{E}}
\def\cH{\mathcal{H}}
\def\cI{\mathcal{I}}
\def\cN{\mathcal{N}}
\def\cP{\mathcal{P}}
\def\cS{\mathcal{S}}
\def\cO{\mathcal{O}}
\def\cU{\mathcal{U}}
\def\al{\alpha}
\def\be{\beta}
\def\ga{\gamma}
\def\de{\delta}
\def\ep{\epsilon}
\def\ve{\varepsilon}
\def\ze{\zeta}
\def\et{\eta}
\def\th{\theta}
\def\vt{\vartheta}
\def\io{\iota}
\def\ka{\kappa}
\def\la{\lambda}
\def\vpi{\varpi}
\def\rh{\rho}
\def\vr{\varrho}
\def\si{\sigma}
\def\vs{\varsigma}
\def\ta{\tau}
\def\up{\upsilon}
\def\ph{\phi}
\def\vp{\varphi}
\def\ch{\chi}
\def\ps{\psi}
\def\om{\omega}
\def\Ga{\Gamma}
\def\De{\Delta}
\def\Th{\Theta}
\def\La{\Lambda}
\def\Si{\Sigma}
\def\Up{\Upsilon}
\def\Ph{\Phi}
\def\Ps{\Psi}
\def\Om{\Omega}

\newcommand{\cuno}{\text{\ding{192}}}
\newcommand{\cdos}{\text{\ding{193}}}
\newcommand{\ctre}{\text{\ding{194}}}
\newcommand{\ccua}{\text{\ding{195}}}
\newcommand{\ccin}{\text{\ding{196}}}

\def\pt#1{\phantom{#1}}
\def\prt{\partial}
\def\half{\frac{1}{2}}
\def\halfs{\frac{1}{\sqrt{2}}}
\def\lsim{\mathrel{\rlap{\lower4pt\hbox{\hskip1pt$\sim$}}
		\raise1pt\hbox{$<$}}}
\def\gsim{\mathrel{\rlap{\lower4pt\hbox{\hskip1pt$\sim$}}
		\raise1pt\hbox{$>$}}}
\def\Re{\hbox{Re}\,}
\def\Im{\hbox{Im}\,}
\def\Arg{\hbox{Arg}\,}

\definecolor{astral}{RGB}{0,153,153}
\definecolor{azulclaro}{RGB}{153,153,255}
\definecolor{cinves}{RGB}{128,153,229}
\colorlet{guinda}{black}
\chapterfont{\color{black}}
\sectionfont{\color{black}}
\subsectionfont{\color{black}}

\newcommand{\art}{\noindent\fcolorbox{astral}{white}{\textcolor{black}{\textbf{Artículo}}}\hspace{1mm}}
\newcommand{\vid}{\noindent\fcolorbox{black}{white}{\textcolor{guinda}{\textbf{Video}}}\hspace{1mm}}
\newcommand{\lib}{\noindent\fcolorbox{guinda}{white}{\textcolor{guinda}{\textbf{Libro}}}\hspace{1mm}}
\providecommand{\caja}[1]{\noindent\fcolorbox{guinda}{white}{\textcolor{guinda}{\textbf{#1}}}\hspace{0mm}}

\renewcommand{\labelitemi}{\textbullet}% bullet
\renewcommand{\labelitemii}{\textendash}% dash
\renewcommand{\labelitemiii}{\textperiodcentered}% dash
\setlist[itemize]{noitemsep}
\setlist[enumerate]{noitemsep}

%Inicia para vinculos
\usepackage[pdftex, plainpages = false, pdfpagelabels, 
                 pdfpagelayout = SinglePage,
                 bookmarks,
                 bookmarksopen = true,
                 bookmarksnumbered = true,
                 breaklinks = true,
                 linktocpage,
                 colorlinks = true,
                 linkcolor = blue,
                 urlcolor  = blue,
                 citecolor = red,
                 anchorcolor = green,
                 hyperindex = true,
                 hyperfigures
                 ]{hyperref} 
\pdfcompresslevel=9
\pdfstringdefDisableCommands{%
	\renewcommand*{\bm}[1]{#1}%
	% any other necessary redefinitions 
}
\hypersetup{
	%    bookmarks=true,         % show bookmarks bar?
	unicode=false,          % non-Latin characters in Acrobat’s bookmarks
	pdffitwindow=false,     % window fit to page when opened
	pdfstartview={FitH},    % fits the width of the page to the window
	pdftitle={Inestabilidades en sistemas ac\'usticos y \'opticos},    % title
	pdfauthor={Juan David Rinc\'on},     % author
	pdfsubject={Radiaci\'on de Hawking an\'aloga},   % subject of the document
	pdfcreator={Juan David Rinc\'on},   % creator of the document
	pdfproducer={Juan David Rinc\'on}, % producer of the document
	pdfkeywords={tesis, radiaci\'on de Hawking,Becs, fibras \'opticas, español, spanish, David, Bermudez}, % list of keywords
	pdfnewwindow=true,      % links in new window
	colorlinks=true,       % false: boxed links; true: colored links
	linkcolor=cinves,          % color of internal links
	citecolor=astral,        % color of links to bibliography
	filecolor=astral,      % color of file links
	urlcolor=blue           % color of external links
}
%Termina para vinculos

% Inicia formato de inicio de cap\'itulo
\makeatletter
\def\thickhrulefill{\leavevmode \leaders \hrule height 1ex \hfill \kern \z@}
\def\@makechapterhead#1{%
  \vspace*{10\p@}%
  {\parindent \z@ \raggedleft \reset@font
            \scshape \textcolor{guinda}{\@chapapp{} \thechapter}
        \par\nobreak
        \interlinepenalty\@M
    \Huge \bfseries \textcolor{black}{#1}\par\nobreak
    %\vspace*{1\p@}%
    \hrulefill
    \par\nobreak
    \vskip 100\p@
  }}
\def\@makeschapterhead#1{%
  \vspace*{10\p@}%
  {\parindent \z@ \raggedleft \reset@font
            \scshape \textcolor{guinda}{\vphantom{\@chapapp{} \thechapter}}
        \par\nobreak
        \interlinepenalty\@M
    \Huge \bfseries \textcolor{black}{#1}\par\nobreak
    %\vspace*{1\p@}%
    \hrulefill
    \par\nobreak
    \vskip 100\p@
  }}
% Termina formato de inicio de cap\'itulo

%Inicia formato pregunta
\newcounter{preg1}
\newcounter{preg2}
\newcounter{preg3}
\newcounter{preg4}
\newcounter{preg5}
\newcounter{preg6}
\newcounter{preg7}
\newcounter{preg8}
\newcounter{preg9}
\newcounter{preg10}

\newenvironment{pregunta}[1]
{\begin{enumerate}[resume=#1,noitemsep,label={\thechapter.\arabic*.}]\itshape}
{\end{enumerate}}

\tcolorboxenvironment{pregunta}{
	empty, before skip=3mm,after skip=3mm,
	coltitle=guinda,fonttitle=\bfseries,
	title={\hspace{0mm}Preguntas:},	% Attaching a box requires an overlay
	attach boxed title to top left,% Ensures proper line breaking in longer titles
	minipage boxed title,% (boxed title style requires an overlay)
	boxed title style={empty,size=minimal,toprule=0pt,top=4pt,left=3mm,overlay={}},
	coltitle=guinda,fonttitle=\bfseries,
	before=\par\medskip\noindent,parbox=false,boxsep=0mm,left=0mm,right=0mm,top=0mm,breakable,pad at break=0mm,
	before upper=\csname @totalleftmargin\endcsname0pt, % Use instead of parbox=true. This ensures parskip is inherited by box.
	% Handles box when it exists on one page only
	overlay unbroken={\draw[guinda,line width=1pt] ([xshift=-0pt]title.north west) -- ([xshift=-0pt]frame.south west); },
	% Handles multipage box: first, middle and last pages
	overlay first={\draw[guinda,line width=1pt] ([xshift=-0pt]title.north west) -- ([xshift=-0pt]frame.south west); },
	overlay middle={\draw[guinda,line width=1pt] ([xshift=-0pt]frame.north west) -- ([xshift=-0pt]frame.south west); },
	overlay last={\draw[guinda,line width=1pt] ([xshift=-0pt]frame.north west) -- ([xshift=-0pt]frame.south west); },
}
%Termina formato pregunta