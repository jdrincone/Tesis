\chapter*{Resumen}
En este trabajo se muestra que la radiaci\'{o}n de Hawking resonante en la configuraci\'{o}n de l\'{a}ser de agujeros negros (BHL) es producto de un conjunto discreto de inestabilidades, tanto si el sistema es un condensado de Bose-Einstein (BECs) donde la radiaci\'{o}n son fonones, como si el sistema son pulsos de luz propag\'{a}ndose en una fibra \'{o}ptica donde la radiaci\'{o}n son fotones. Para obtener este resultado, describimos las propiedades de un campo cu\'{a}ntico al propagarse en un espacio-tiempo curvo como el generado por un agujero negro. Este tipo de curvatura posee una regi\'{o}n de no retorno que define un horizonte de eventos y la radiaci\'{o}n que escapa de \'{e}sta se conoce como radiaci\'{o}n de Hawking (HR).\\

Mostramos c\'omo la propagaci\'{o}n de una fluctuaci\'{o}n sobre un fluido en movimiento es equivalente a la propagaci\'{o}n de un campo escalar en un espacio-tiempo curvo. Este resultado nos permite estudiar la HR en experimentos de laboratorio en materia condensada y en \'{o}ptica. Luego estudiamos la relaci\'{o}n de dispersi\'{o}n que debe cumplir una fluctuaci\'{o}n del estado base de un sistema bos\'{o}nico como los BECs a partir de la linealizaci\'{o}n de la ecuaci\'{o}n de Gross-Pitaevkii. Demostramos que la din\'{a}mica de la fluctuaci\'{o}n que se propaga sobre el fluido en movimiento puede ser analizada por teor\'{i}a de inestabilidades. Despu\'{e}s introducimos el BHL en el contexto ac\'{u}stico y analizamos la din\'{a}mica de la fluctuaci\'{o}n por teor\'{i}a de inestabilidades de la manera que fue introducida en \cite{2018Bermudez}.\\

M\'{a}s tarde, implementamos un formalismo similar para el caso de sistemas \'{o}pticos, donde el medio en movimiento ser\'{a} creado a partir de la interacci\'{o}n de pulsos \'{o}pticos intensos con materiales no lineales y dispersivos. Estudiamos la relaci\'{o}n de dispersi\'{o}n que debe seguir una fluctuaci\'{o}n que se propaga sobre un pulso de luz que crea el medio en movimiento, esto se logra linealizando la ecuaci\'{o}n no lineal de Schr\"{o}dinger (NLSE). Analizamos igualmente la din\'{a}mica de la fluctuaci\'{o}n en la configuraci\'{o}n del l\'{a}ser \'{o}ptico de agujeros negros (OBHL) por teor\'{i}a de inestabilidades. Concluimos que la configuraci\'{o}n de OBHL permite confinar luz con luz de forma similar a una gu\'{i}a de ondas temporal y que tambi\'{e}n se amplifica para producir la radiaci\'{o}n de Hawking resonante.